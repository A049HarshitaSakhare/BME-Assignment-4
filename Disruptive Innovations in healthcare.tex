\documentclass[12pt,a4paper]{article}

\usepackage{tikz}
\usetikzlibrary{calc}

\usepackage{graphicx}
\graphicspath{{Images/}}


\usepackage[T1]{fontenc}
\usepackage{tgbonum}

\begin{document}


\begin{tikzpicture}
[remember picture, overlay]  \draw[line width=5pt]  ($(current page.north west)+(0.3in,-0.3in)$)  rectangle ($(current page.south east)+(-0.3in,0.3in)$);


\end{tikzpicture}

\begin{center}
\textbf{NATIONAL INSTITUTE OF TECHNOLOGY RAIPUR}
\end{center}

\begin{figure}[h]
\centering
\includegraphics[scale=0.2]{NITRR.jpg}
\end{figure}

\hspace{1cm}

\begin{center}
\textit{\large Submitted By:- Harshita Upendra Sakhare}
\end{center}


\hspace{1cm}


\begin{center}
\textsl{Roll.No:- 21111049}
\end{center}

\hspace{1cm}


\begin{center}
\textbf{\large Basics of Biomedical Engineering}
\end{center}

\begin{center}
\textbf{\large Assignment-4}
\end{center}

\hspace{1cm}

\begin{center}
\textsc{WRITE UP ON DISRUPTIVE INNOVATIONS IN HEALTHCARE}
\end{center}



\hspace{1cm}

\begin{center}
\textit{\large Submitted To:- Prof. Saurabh Gupta}
\end{center}

\hspace{1cm}

\begin{center}
\textsl{Date of Submission:- 9 February 2022 }
\end{center}

\clearpage
\tableofcontents
\clearpage

\section{Acknowledgement}
\hspace{1cm}
In successful completion of my assignment on DISRUPTIVE\\INNOVATIONS IN HEALTHCARE, I would like to thank my\\Professor. Saurabh Gupta Lecturer of Biomedical Engineering, who has guided and assisted me to complete the assignment. Without his support I would not have finished the assignment within time.


I  would also like to take this opportunity to thank my friends and family members, without them this assignment could not have been completed in a short duration.


\clearpage

\section{Introduction}
\hspace{1cm}
The concept of disruptive innovation has attracted significant attention since it appeared in Clayton Christensen’s influential book “The Innovator’s Dilemma.” Yet its meaning is often misunderstood or misapplied. More than a decade after the book’s publication, Christensen revisited the theory in the pages of the Harvard Business Review. He and his co-authors emphasized that disruptive innovation begins with products creating low-end or new-market footholds that relentlessly move upmarket, eventually displacing established competitors. In other words, disruption starts when a company identifies and satisfies an unmet (and sometimes latent) need.


More recently, Netflix is an example of disruptive innovation. Catering at first to a niche group of consumers with its web interface and mail-order delivery, Netflix didn’t appear to challenge Blockbuster, the industry leader in movie rentals. However, the advent of streaming technology enabled Netflix to completely re-think the business. Netflix’s ability to pivot quickly, along with the convenience, ease-of-use, and low price point of its offering, allowed it to quickly unseat Blockbuster.

\hspace{2cm}


\begin{figure}[h]
\centering
\includegraphics[scale=0.3]{disruptive.jpg}
\end{figure}





\clearpage



\section{Disruptive Innovations In Healthcare}

 Disruptive innovations are those that cause radical change and often result in new leaders in the field. They overturn the usual way of doing things to such an extent that they have a ripple effect throughout the industry. The following nine examples of disruptive innovations in healthcare are centered on technology, customer-centric care, and third-party advancements.

\hspace{5cm}

\begin{figure}[h]
\centering
\includegraphics[scale=0.2]{innovation HC.jpg}
\end{figure}


\hspace{5cm}


\begin{itemize}
\item \textbf{Technology}

Technology is the biggest driver of many disruptive innovations in healthcare since every aspect of healthcare is dependent on some form of tech. From wearables and mobile phone apps to big data and artificial intelligence (AI) use in diagnosis, any new technology could potentially shake up healthcare.
\end{itemize}

\clearpage

\section{Examples of Disruptive Innovations In\\Healthcare}


\hspace{1cm}

\textbf {1. Consumer devices, wearables, and apps}

\hspace{1cm}
     
In the past, a patient could get only biometric data about their pulse, heart rate, blood oxygen, and blood pressure when they went to the doctor’s office. Now, consumers take charge of their own health journey, using data gathered from their Fitbits, smartwatches, and mobile phone fitness apps. Physicians can use the data gathered from these wearables to make treatment decisions, although the vast amount of personal information collected by these apps has led to legal and ethical concerns over data privacy.

\hspace{5cm}

\begin{figure}[h]
\centering
\includegraphics[scale=0.3]{consumers.jpg}
\caption{Consumer devices,wearables and apps}
\end{figure}



\clearpage




\textbf{2. AI and machine learning}

\hspace{1cm}

            
AI applications can manage patient intake and scheduling as well as billing. Chatbots answer patient questions. With natural language processing capabilities, AI can collate and analyze survey responses. AI will probably increase in use as a way to bring down healthcare costs and let doctors and staff focus on patient care. Healthcare leaders must be knowledgeable about the issues surrounding database management and patient privacy. 

\hspace{5cm}

\begin{figure}[h]
\centering
\includegraphics[scale=0.3]{AI.jpg}
\caption{AI and Machine learning}
\end{figure}



\textbf{3. Blockchain}

\hspace{1cm}
           
Blockchain is a database technology that uses encryption and other security measures to store data and link it in a way that enhances security and usability. This innovation facilitates many aspects of healthcare, including patient records, supply and distribution, and research. Tech startups have entered the healthcare sector with blockchain applications that have changed how providers use medical data. 

\begin{figure}[h]
\centering
\includegraphics[scale=0.4]{blockchain.jpg}
\caption{Blockchain in Healthcare}
\end{figure}


\clearpage

\textbf{4.IoT}

\hspace{1cm}

 IoT enables healthcare professionals to be more watchful and connect with the patients proactively.  IoT devices tagged with sensors are used for tracking real time location of medical equipment like wheelchairs, defibrillators, nebulizers, oxygen pumps and other monitoring equipment. 
 
 \hspace{5cm}
 
\begin{figure}[h]
\centering
\includegraphics[scale=1]{iot2.jpg}
\caption{IoT in Healthcare}
\end{figure}
 
\hspace{5cm}

\begin{itemize}
\item \textbf{Consumer-centered care}


Many examples of disruptive innovations in healthcare pertain to consumer-centered care. With the increasing consumerization of healthcare, the patient-healthcare provider relationship has also undergone radical change. In this arena, the combination of technology and public policy has transformed how patients access healthcare and interact with their healthcare providers.  
\end{itemize}

\clearpage

\textbf{5. Electronic health records and big data}

\hspace{1cm}
             
Electronic health records (EHRs) have been a growing part of patient care since the adoption of the Affordable Care Act. The massive amount of EHR data goes far beyond patient health records, however, and can be used to conduct research, improve care, build AI applications, and create new business opportunities. Therefore, healthcare providers have to be aware of the issues surrounding EHR security.

\hspace{5cm}


\begin{figure}[h]
\centering
\includegraphics[scale=0.5]{EHR.jpg}
\caption{EHR and big data}
\end{figure}

\clearpage

\textbf{6. Telemedicine}

\hspace{1cm}
          
COVID-19 has undoubtedly accelerated the delivery of telemedicine, and experts affirm that telemedicine is here to stay. It’s effective, doctors will be reimbursed for a telehealth consultation, and many patients prefer it. However, telemedicine is highly dependent on internet access, and some areas of the U.S. still have poor connectivity.

\begin{figure}[h]
\centering
\includegraphics[scale=0.2]{TM.jpg}
\caption{Telemedicine in Healthcare}
\end{figure}

\clearpage

\textbf{7. Patient rights}

\hspace{1cm}

EHR data security, billing transparency, and access to medical records are all parts of a major shift in healthcare that ensures that patient receive all the information they need to make informed decisions about there care. As of early 2021, hospitals must make their prices more transparent, per the centres for medicare and medicaid services(CMS). Other upcoming reforms include introduction of online pricing tools so patients can see there out of pocket costs.

\hspace{5cm}


\begin{figure}[h]
\centering
\includegraphics[scale=0.3]{PR2.jpg}
\caption{Patient rights}
\end{figure}

\hspace{5cm}


\begin{itemize}

\item \textbf{Third-party advancements}

As with blockchain, third-party firms have altered every aspect of healthcare. Besides tech companies, the biggest disruptors of healthcare are retail giants. They acquire manufacturers and wholesalers and disrupt the supply chain. 
\end{itemize}

\clearpage


\textbf{8. Retail competition}

\hspace{1cm}

           
In 2019, Walmart formed Walmart Health, freestanding clinics that provide primary and urgent care. The same year, Amazon bought the online pharmacy PillPack, setting itself up to move into the pharmaceutical retail market and potentially disrupt the pharmacy benefits management market. In 2018, CVS acquired Aetna, moving from retail into health plans. All of these moves create new giants in the industry, changing the way healthcare operates. 


\hspace{5cm}

\begin{figure}[h]
\centering
\includegraphics[scale=0.5]{RC.jpg}
\caption{Retail Competition}
\end{figure}


\textbf{9. Public policy}

\hspace{1cm}
         
CMS announced that public health plans will have to make pricing data available to all in 2022, with the goal of driving innovation and providing consumers with pricing comparison tools. Researchers, entrepreneurs, and developers will be able to access this data to build new tools for patients. 

\hspace{5cm}

\begin{figure}[h]
\centering
\includegraphics[scale=0.3]{PP.jpg}
\caption{Public Policy}
\end{figure}
 

\section{Conclusion}

Disruptive innovations is a term that has diffused into the healthcare industry,but there is widespread ambiguity in the use of the term. It may have become a victim of its own mainstream success. Poor identification can lead to poor understanding of the characteristics and potential of and innovation. This in turn can contribute to delay in its translation into tangible economic and health outcomes-based benefits because we fail to understand the potential barriers to adoption and ways to overcome them.


\clearpage

\section{Reference}

\begin{itemize}
\item Wikipidea
\item innovations.bmj.com
\item dhge.org
\end{itemize}









\end{document}